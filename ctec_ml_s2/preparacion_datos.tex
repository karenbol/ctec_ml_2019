\documentclass[]{article}
\usepackage{lmodern}
\usepackage{amssymb,amsmath}
\usepackage{ifxetex,ifluatex}
\usepackage{fixltx2e} % provides \textsubscript
\ifnum 0\ifxetex 1\fi\ifluatex 1\fi=0 % if pdftex
  \usepackage[T1]{fontenc}
  \usepackage[utf8]{inputenc}
\else % if luatex or xelatex
  \ifxetex
    \usepackage{mathspec}
  \else
    \usepackage{fontspec}
  \fi
  \defaultfontfeatures{Ligatures=TeX,Scale=MatchLowercase}
\fi
% use upquote if available, for straight quotes in verbatim environments
\IfFileExists{upquote.sty}{\usepackage{upquote}}{}
% use microtype if available
\IfFileExists{microtype.sty}{%
\usepackage{microtype}
\UseMicrotypeSet[protrusion]{basicmath} % disable protrusion for tt fonts
}{}
\usepackage[margin=1in]{geometry}
\usepackage{hyperref}
\hypersetup{unicode=true,
            pdftitle={Preparación de los datos},
            pdfborder={0 0 0},
            breaklinks=true}
\urlstyle{same}  % don't use monospace font for urls
\usepackage{color}
\usepackage{fancyvrb}
\newcommand{\VerbBar}{|}
\newcommand{\VERB}{\Verb[commandchars=\\\{\}]}
\DefineVerbatimEnvironment{Highlighting}{Verbatim}{commandchars=\\\{\}}
% Add ',fontsize=\small' for more characters per line
\usepackage{framed}
\definecolor{shadecolor}{RGB}{248,248,248}
\newenvironment{Shaded}{\begin{snugshade}}{\end{snugshade}}
\newcommand{\AlertTok}[1]{\textcolor[rgb]{0.94,0.16,0.16}{#1}}
\newcommand{\AnnotationTok}[1]{\textcolor[rgb]{0.56,0.35,0.01}{\textbf{\textit{#1}}}}
\newcommand{\AttributeTok}[1]{\textcolor[rgb]{0.77,0.63,0.00}{#1}}
\newcommand{\BaseNTok}[1]{\textcolor[rgb]{0.00,0.00,0.81}{#1}}
\newcommand{\BuiltInTok}[1]{#1}
\newcommand{\CharTok}[1]{\textcolor[rgb]{0.31,0.60,0.02}{#1}}
\newcommand{\CommentTok}[1]{\textcolor[rgb]{0.56,0.35,0.01}{\textit{#1}}}
\newcommand{\CommentVarTok}[1]{\textcolor[rgb]{0.56,0.35,0.01}{\textbf{\textit{#1}}}}
\newcommand{\ConstantTok}[1]{\textcolor[rgb]{0.00,0.00,0.00}{#1}}
\newcommand{\ControlFlowTok}[1]{\textcolor[rgb]{0.13,0.29,0.53}{\textbf{#1}}}
\newcommand{\DataTypeTok}[1]{\textcolor[rgb]{0.13,0.29,0.53}{#1}}
\newcommand{\DecValTok}[1]{\textcolor[rgb]{0.00,0.00,0.81}{#1}}
\newcommand{\DocumentationTok}[1]{\textcolor[rgb]{0.56,0.35,0.01}{\textbf{\textit{#1}}}}
\newcommand{\ErrorTok}[1]{\textcolor[rgb]{0.64,0.00,0.00}{\textbf{#1}}}
\newcommand{\ExtensionTok}[1]{#1}
\newcommand{\FloatTok}[1]{\textcolor[rgb]{0.00,0.00,0.81}{#1}}
\newcommand{\FunctionTok}[1]{\textcolor[rgb]{0.00,0.00,0.00}{#1}}
\newcommand{\ImportTok}[1]{#1}
\newcommand{\InformationTok}[1]{\textcolor[rgb]{0.56,0.35,0.01}{\textbf{\textit{#1}}}}
\newcommand{\KeywordTok}[1]{\textcolor[rgb]{0.13,0.29,0.53}{\textbf{#1}}}
\newcommand{\NormalTok}[1]{#1}
\newcommand{\OperatorTok}[1]{\textcolor[rgb]{0.81,0.36,0.00}{\textbf{#1}}}
\newcommand{\OtherTok}[1]{\textcolor[rgb]{0.56,0.35,0.01}{#1}}
\newcommand{\PreprocessorTok}[1]{\textcolor[rgb]{0.56,0.35,0.01}{\textit{#1}}}
\newcommand{\RegionMarkerTok}[1]{#1}
\newcommand{\SpecialCharTok}[1]{\textcolor[rgb]{0.00,0.00,0.00}{#1}}
\newcommand{\SpecialStringTok}[1]{\textcolor[rgb]{0.31,0.60,0.02}{#1}}
\newcommand{\StringTok}[1]{\textcolor[rgb]{0.31,0.60,0.02}{#1}}
\newcommand{\VariableTok}[1]{\textcolor[rgb]{0.00,0.00,0.00}{#1}}
\newcommand{\VerbatimStringTok}[1]{\textcolor[rgb]{0.31,0.60,0.02}{#1}}
\newcommand{\WarningTok}[1]{\textcolor[rgb]{0.56,0.35,0.01}{\textbf{\textit{#1}}}}
\usepackage{graphicx,grffile}
\makeatletter
\def\maxwidth{\ifdim\Gin@nat@width>\linewidth\linewidth\else\Gin@nat@width\fi}
\def\maxheight{\ifdim\Gin@nat@height>\textheight\textheight\else\Gin@nat@height\fi}
\makeatother
% Scale images if necessary, so that they will not overflow the page
% margins by default, and it is still possible to overwrite the defaults
% using explicit options in \includegraphics[width, height, ...]{}
\setkeys{Gin}{width=\maxwidth,height=\maxheight,keepaspectratio}
\IfFileExists{parskip.sty}{%
\usepackage{parskip}
}{% else
\setlength{\parindent}{0pt}
\setlength{\parskip}{6pt plus 2pt minus 1pt}
}
\setlength{\emergencystretch}{3em}  % prevent overfull lines
\providecommand{\tightlist}{%
  \setlength{\itemsep}{0pt}\setlength{\parskip}{0pt}}
\setcounter{secnumdepth}{0}
% Redefines (sub)paragraphs to behave more like sections
\ifx\paragraph\undefined\else
\let\oldparagraph\paragraph
\renewcommand{\paragraph}[1]{\oldparagraph{#1}\mbox{}}
\fi
\ifx\subparagraph\undefined\else
\let\oldsubparagraph\subparagraph
\renewcommand{\subparagraph}[1]{\oldsubparagraph{#1}\mbox{}}
\fi

%%% Use protect on footnotes to avoid problems with footnotes in titles
\let\rmarkdownfootnote\footnote%
\def\footnote{\protect\rmarkdownfootnote}

%%% Change title format to be more compact
\usepackage{titling}

% Create subtitle command for use in maketitle
\providecommand{\subtitle}[1]{
  \posttitle{
    \begin{center}\large#1\end{center}
    }
}

\setlength{\droptitle}{-2em}

  \title{Preparación de los datos}
    \pretitle{\vspace{\droptitle}\centering\huge}
  \posttitle{\par}
    \author{}
    \preauthor{}\postauthor{}
    \date{}
    \predate{}\postdate{}
  

\begin{document}
\maketitle

\hypertarget{breve-introduccion-a-ggpairs}{%
\section{Breve introducción a
ggpairs}\label{breve-introduccion-a-ggpairs}}

\texttt{ggpairs()} es una función de la librería GGally que permite
crear matrices de gráficos y es realmente útil en la etapa de análisis y
exploración de los datos. Pueden encontrar más información sobre
\texttt{ggpairs()} y GGally en el siguiente enlace
\url{https://ggobi.github.io/ggally/\#ggally}

El siguiente ejemplo muestra el gráfico de las correlaciones y
densidades de algunos atributos del dataset de ventas de casas.

\begin{Shaded}
\begin{Highlighting}[]
\KeywordTok{library}\NormalTok{(GGally)}
\end{Highlighting}
\end{Shaded}

\begin{verbatim}
## Loading required package: ggplot2
\end{verbatim}

\begin{verbatim}
## Registered S3 method overwritten by 'GGally':
##   method from   
##   +.gg   ggplot2
\end{verbatim}

\begin{Shaded}
\begin{Highlighting}[]
\KeywordTok{library}\NormalTok{(ggplot2)}

\NormalTok{casas <-}\StringTok{ }\KeywordTok{read.csv}\NormalTok{(}\StringTok{'kc_house_data.csv'}\NormalTok{, }\DataTypeTok{header =}\NormalTok{ T, }\DataTypeTok{na.strings =} \StringTok{'?'}\NormalTok{)}
\KeywordTok{ggpairs}\NormalTok{(casas, }\DataTypeTok{columns =} \DecValTok{3}\OperatorTok{:}\DecValTok{7}\NormalTok{)}
\end{Highlighting}
\end{Shaded}

\includegraphics{preparacion_datos_files/figure-latex/unnamed-chunk-1-1.pdf}

\hypertarget{limpieza-y-transformacion-de-los-datos}{%
\section{Limpieza y transformación de los
datos}\label{limpieza-y-transformacion-de-los-datos}}

A continuación utilizando el dataset de venta de casas veremos algunas
funcionalidades de R para la transformación de datos.

\hypertarget{deteccion-de-valores-faltantes-y-transformaciones}{%
\subsubsection{Detección de valores faltantes y
transformaciones}\label{deteccion-de-valores-faltantes-y-transformaciones}}

La función \texttt{is.na} nos permite detectar los valores faltantes en
un dataset.

El siguiente código muestra el número de elementos faltantes en todo un
dataset

\texttt{sum(is.na(dataset))}

\begin{Shaded}
\begin{Highlighting}[]
\KeywordTok{sum}\NormalTok{(}\KeywordTok{is.na}\NormalTok{(casas))}
\end{Highlighting}
\end{Shaded}

\begin{verbatim}
## [1] 0
\end{verbatim}

También podemos buscar valores faltantes en columnas especificas

``

\begin{Shaded}
\begin{Highlighting}[]
\KeywordTok{sum}\NormalTok{(}\KeywordTok{is.na}\NormalTok{(casas}\OperatorTok{$}\NormalTok{view))}
\end{Highlighting}
\end{Shaded}

\begin{verbatim}
## [1] 0
\end{verbatim}

R posee funciones para modificar dataframes las cuales podemos
aprovechar para el tratamiento de valores faltantes y transformaciones

Podemos asignar valores nuevos a toda una columna.

\texttt{dataframe\$columna\ \textless{}-\ 0}

\texttt{dataframe\$columna\ \textless{}-\ funcion(dataframe\$columna)}

O podemos hacerlo solo en los campos faltantes.

\texttt{dataframe\$columna{[}is.na(dataframe\$columna){]}\ \textless{}-\ 0}

\texttt{dataframe\$columna{[}is.na(dataframe\$columna){]}\ \textless{}-\ funcion(dataframe\$columna)}

\hypertarget{ejercicios-sobre-transformaciones}{%
\section{Ejercicios sobre
transformaciones}\label{ejercicios-sobre-transformaciones}}

\hypertarget{correcion-de-valores-inconsistentes}{%
\subsubsection{1. Correción de valores
inconsistentes}\label{correcion-de-valores-inconsistentes}}

\begin{Shaded}
\begin{Highlighting}[]
\NormalTok{casas}\OperatorTok{$}\NormalTok{bathrooms[}\DecValTok{1}\OperatorTok{:}\DecValTok{10}\NormalTok{]}
\end{Highlighting}
\end{Shaded}

\begin{verbatim}
##  [1] 2.25 1.00 4.50 1.50 1.00 1.75 2.00 3.00 2.00 1.75
\end{verbatim}

Como podemos observar la columna bathrooms posee valores inconsistentes
ya que no tiene sentido que una casa tenga 4.5 baños.

Modifique la columna bathrooms para que solo posea valores enteros

\begin{Shaded}
\begin{Highlighting}[]
\CommentTok{# Escriba su código aqui}
\CommentTok{# metodo floor redondea hacia abajo}
\NormalTok{casas}\OperatorTok{$}\NormalTok{bathrooms <-}\StringTok{ }\KeywordTok{floor}\NormalTok{(casas}\OperatorTok{$}\NormalTok{bathrooms)}
\NormalTok{casas}\OperatorTok{$}\NormalTok{bathrooms[}\DecValTok{1}\OperatorTok{:}\DecValTok{10}\NormalTok{]}
\end{Highlighting}
\end{Shaded}

\begin{verbatim}
##  [1] 2 1 4 1 1 1 2 3 2 1
\end{verbatim}

\hypertarget{llevar-los-datos-a-distribucion-normal.}{%
\subsubsection{2. Llevar los datos a distribución
normal.}\label{llevar-los-datos-a-distribucion-normal.}}

Modifique la columna price del dataset de precios de casas para que siga
una distribución normal. Pista: utilice la función \texttt{log10}

\begin{Shaded}
\begin{Highlighting}[]
\CommentTok{# Precios antes de la modificación}
\KeywordTok{hist}\NormalTok{(casas}\OperatorTok{$}\NormalTok{price, }\DataTypeTok{main =} \StringTok{"Distribucion de Precios antes de aplicar log10"}\NormalTok{)}
\end{Highlighting}
\end{Shaded}

\includegraphics{preparacion_datos_files/figure-latex/unnamed-chunk-6-1.pdf}

\begin{Shaded}
\begin{Highlighting}[]
\CommentTok{# Escriba su codigo aqui}
\NormalTok{casas}\OperatorTok{$}\NormalTok{price <-}\StringTok{ }\KeywordTok{log10}\NormalTok{(casas}\OperatorTok{$}\NormalTok{price)}
\CommentTok{# Precios normalmente distribuidos}
\KeywordTok{hist}\NormalTok{(casas}\OperatorTok{$}\NormalTok{price, }\DataTypeTok{main =} \StringTok{"Distribucion de Precios luego de aplicar log10"}\NormalTok{)}
\end{Highlighting}
\end{Shaded}

\includegraphics{preparacion_datos_files/figure-latex/unnamed-chunk-6-2.pdf}

\begin{Shaded}
\begin{Highlighting}[]
\CommentTok{# Como se ve ahora el valor de los precios?}
\NormalTok{casas}\OperatorTok{$}\NormalTok{price[}\DecValTok{1}\OperatorTok{:}\DecValTok{10}\NormalTok{]}
\end{Highlighting}
\end{Shaded}

\begin{verbatim}
##  [1] 5.730782 5.255273 6.088136 5.465160 5.491362 5.602060 5.724276
##  [8] 5.812913 5.596597 5.585461
\end{verbatim}

Note que ahora los precios ya no se agrupan a la izquierda.

\hypertarget{imputar-datos-faltantes}{%
\subsubsection{3. Imputar datos
faltantes}\label{imputar-datos-faltantes}}

Reemplace los valores faltantes de la columna sqft\_living con la
mediana de los valores de esa columna. Pista use la función
\texttt{median}.

Nota: este dataset no posee valores faltantes pero para efectos del
ejercicio esto no nos afecta ya que el código sería igual.

\begin{Shaded}
\begin{Highlighting}[]
\NormalTok{sqft_living_median =}\StringTok{ }\KeywordTok{median}\NormalTok{(casas}\OperatorTok{$}\NormalTok{sqft_living)}

\NormalTok{casas}\OperatorTok{$}\NormalTok{sqft_living[}\KeywordTok{is.na}\NormalTok{(casas}\OperatorTok{$}\NormalTok{sqft_living)] <-}\StringTok{ }\NormalTok{sqft_living_median}
\end{Highlighting}
\end{Shaded}

\hypertarget{escalado}{%
\subsubsection{4. Escalado}\label{escalado}}

Un proceso común en esta etapa es escalar los atributos para que los
valores queden en el rango de 0-1.

La siguiente función realiza el escalado de valores

\begin{Shaded}
\begin{Highlighting}[]
\NormalTok{feature_scaling <-}\StringTok{ }\ControlFlowTok{function}\NormalTok{(x) \{}
\NormalTok{  x_escalado <-}\StringTok{ }\NormalTok{((x }\OperatorTok{-}\StringTok{ }\KeywordTok{min}\NormalTok{(x)) }\OperatorTok{/}\StringTok{ }\NormalTok{(}\KeywordTok{max}\NormalTok{(x) }\OperatorTok{-}\StringTok{ }\KeywordTok{min}\NormalTok{(x)))}
  \KeywordTok{return}\NormalTok{(x_escalado)}
\NormalTok{\}}
\end{Highlighting}
\end{Shaded}

\begin{enumerate}
\def\labelenumi{\arabic{enumi}.}
\tightlist
\item
  Utilice la función \texttt{feature\_scaling} para escalar los valores
  de la columna sqft\_lot
\item
  Compare el resultado con la función \texttt{scale()}
\end{enumerate}

\begin{Shaded}
\begin{Highlighting}[]
\NormalTok{casas}\OperatorTok{$}\NormalTok{sqft_lot <-}\StringTok{ }\KeywordTok{feature_scaling}\NormalTok{(casas}\OperatorTok{$}\NormalTok{sqft_lot)}
\NormalTok{resultado_scale <-}\StringTok{ }\KeywordTok{scale}\NormalTok{(casas}\OperatorTok{$}\NormalTok{sqft_lot)}

\CommentTok{# Las funciones feature_scaling y scale dan resultados diferentes porque aplican cálculos matematicos diferentes.}
\CommentTok{# scale por ejemplo centra los valores restandole a cada uno el promedio de los datos, y los escala dividiendo lo valores centrados}
\CommentTok{# entre su desviacion estandar.}
\end{Highlighting}
\end{Shaded}

\hypertarget{normalizacion-con-la-funcion-z-score}{%
\subsubsection{5. Normalización con la función
z-score}\label{normalizacion-con-la-funcion-z-score}}

Otra función utilizada para transformar atributos se llama z-score y
consiste en restar la media de los datos y dividir por la desviación
estandar.

Escriba la función z\_score y utilicela sobre la columna bathrooms.

Pista: utilice las funciones \texttt{mean}, \texttt{sd}

\begin{Shaded}
\begin{Highlighting}[]
\NormalTok{b_mean =}\StringTok{ }\KeywordTok{mean}\NormalTok{(casas}\OperatorTok{$}\NormalTok{bathrooms)}
\end{Highlighting}
\end{Shaded}

\begin{Shaded}
\begin{Highlighting}[]
\NormalTok{b_sd =}\StringTok{ }\KeywordTok{sd}\NormalTok{(casas}\OperatorTok{$}\NormalTok{bathrooms)}
\end{Highlighting}
\end{Shaded}

\begin{Shaded}
\begin{Highlighting}[]
\NormalTok{(}\DecValTok{4} \OperatorTok{-}\StringTok{ }\NormalTok{b_mean) }\OperatorTok{/}\StringTok{ }\NormalTok{b_sd}
\end{Highlighting}
\end{Shaded}

\begin{verbatim}
## [1] 3.125862
\end{verbatim}

\begin{Shaded}
\begin{Highlighting}[]
\CommentTok{#funcion z_score}
\NormalTok{z_score =}\StringTok{ }\ControlFlowTok{function}\NormalTok{(x)\{}
\NormalTok{  data_mean =}\StringTok{ }\KeywordTok{mean}\NormalTok{(x)}
\NormalTok{  data_sd =}\StringTok{ }\KeywordTok{sd}\NormalTok{(x)}
  
\NormalTok{  result =}\StringTok{ }\NormalTok{(x }\OperatorTok{-}\StringTok{ }\NormalTok{data_mean) }\OperatorTok{/}\StringTok{ }\NormalTok{data_sd}
  \KeywordTok{return}\NormalTok{(result)}
\NormalTok{\}}
\end{Highlighting}
\end{Shaded}

\begin{Shaded}
\begin{Highlighting}[]
\CommentTok{# Datos antes de la transformación}
\KeywordTok{hist}\NormalTok{(casas}\OperatorTok{$}\NormalTok{bathrooms, }\DataTypeTok{main =} \StringTok{"Antes de aplicar Z-score"}\NormalTok{)}
\end{Highlighting}
\end{Shaded}

\includegraphics{preparacion_datos_files/figure-latex/unnamed-chunk-14-1.pdf}

\begin{Shaded}
\begin{Highlighting}[]
\NormalTok{casas}\OperatorTok{$}\NormalTok{bathrooms =}\StringTok{ }\KeywordTok{z_score}\NormalTok{(casas}\OperatorTok{$}\NormalTok{bathrooms)}

\CommentTok{# Después de la transformación}
\KeywordTok{hist}\NormalTok{(casas}\OperatorTok{$}\NormalTok{bathrooms, }\DataTypeTok{main =} \StringTok{"Luego de aplicar Z-score"}\NormalTok{)}
\end{Highlighting}
\end{Shaded}

\includegraphics{preparacion_datos_files/figure-latex/unnamed-chunk-14-2.pdf}

Note que ahora la media de los datos se acerca a 0

\begin{Shaded}
\begin{Highlighting}[]
\CommentTok{# cual es la media del set de datos ahora?}
\KeywordTok{mean}\NormalTok{(casas}\OperatorTok{$}\NormalTok{bathrooms)}
\end{Highlighting}
\end{Shaded}

\begin{verbatim}
## [1] 3.647762e-17
\end{verbatim}


\end{document}
