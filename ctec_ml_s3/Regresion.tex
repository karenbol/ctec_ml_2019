\documentclass[]{article}
\usepackage{lmodern}
\usepackage{amssymb,amsmath}
\usepackage{ifxetex,ifluatex}
\usepackage{fixltx2e} % provides \textsubscript
\ifnum 0\ifxetex 1\fi\ifluatex 1\fi=0 % if pdftex
  \usepackage[T1]{fontenc}
  \usepackage[utf8]{inputenc}
\else % if luatex or xelatex
  \ifxetex
    \usepackage{mathspec}
  \else
    \usepackage{fontspec}
  \fi
  \defaultfontfeatures{Ligatures=TeX,Scale=MatchLowercase}
\fi
% use upquote if available, for straight quotes in verbatim environments
\IfFileExists{upquote.sty}{\usepackage{upquote}}{}
% use microtype if available
\IfFileExists{microtype.sty}{%
\usepackage{microtype}
\UseMicrotypeSet[protrusion]{basicmath} % disable protrusion for tt fonts
}{}
\usepackage[margin=1in]{geometry}
\usepackage{hyperref}
\hypersetup{unicode=true,
            pdftitle={Regresion},
            pdfborder={0 0 0},
            breaklinks=true}
\urlstyle{same}  % don't use monospace font for urls
\usepackage{color}
\usepackage{fancyvrb}
\newcommand{\VerbBar}{|}
\newcommand{\VERB}{\Verb[commandchars=\\\{\}]}
\DefineVerbatimEnvironment{Highlighting}{Verbatim}{commandchars=\\\{\}}
% Add ',fontsize=\small' for more characters per line
\usepackage{framed}
\definecolor{shadecolor}{RGB}{248,248,248}
\newenvironment{Shaded}{\begin{snugshade}}{\end{snugshade}}
\newcommand{\AlertTok}[1]{\textcolor[rgb]{0.94,0.16,0.16}{#1}}
\newcommand{\AnnotationTok}[1]{\textcolor[rgb]{0.56,0.35,0.01}{\textbf{\textit{#1}}}}
\newcommand{\AttributeTok}[1]{\textcolor[rgb]{0.77,0.63,0.00}{#1}}
\newcommand{\BaseNTok}[1]{\textcolor[rgb]{0.00,0.00,0.81}{#1}}
\newcommand{\BuiltInTok}[1]{#1}
\newcommand{\CharTok}[1]{\textcolor[rgb]{0.31,0.60,0.02}{#1}}
\newcommand{\CommentTok}[1]{\textcolor[rgb]{0.56,0.35,0.01}{\textit{#1}}}
\newcommand{\CommentVarTok}[1]{\textcolor[rgb]{0.56,0.35,0.01}{\textbf{\textit{#1}}}}
\newcommand{\ConstantTok}[1]{\textcolor[rgb]{0.00,0.00,0.00}{#1}}
\newcommand{\ControlFlowTok}[1]{\textcolor[rgb]{0.13,0.29,0.53}{\textbf{#1}}}
\newcommand{\DataTypeTok}[1]{\textcolor[rgb]{0.13,0.29,0.53}{#1}}
\newcommand{\DecValTok}[1]{\textcolor[rgb]{0.00,0.00,0.81}{#1}}
\newcommand{\DocumentationTok}[1]{\textcolor[rgb]{0.56,0.35,0.01}{\textbf{\textit{#1}}}}
\newcommand{\ErrorTok}[1]{\textcolor[rgb]{0.64,0.00,0.00}{\textbf{#1}}}
\newcommand{\ExtensionTok}[1]{#1}
\newcommand{\FloatTok}[1]{\textcolor[rgb]{0.00,0.00,0.81}{#1}}
\newcommand{\FunctionTok}[1]{\textcolor[rgb]{0.00,0.00,0.00}{#1}}
\newcommand{\ImportTok}[1]{#1}
\newcommand{\InformationTok}[1]{\textcolor[rgb]{0.56,0.35,0.01}{\textbf{\textit{#1}}}}
\newcommand{\KeywordTok}[1]{\textcolor[rgb]{0.13,0.29,0.53}{\textbf{#1}}}
\newcommand{\NormalTok}[1]{#1}
\newcommand{\OperatorTok}[1]{\textcolor[rgb]{0.81,0.36,0.00}{\textbf{#1}}}
\newcommand{\OtherTok}[1]{\textcolor[rgb]{0.56,0.35,0.01}{#1}}
\newcommand{\PreprocessorTok}[1]{\textcolor[rgb]{0.56,0.35,0.01}{\textit{#1}}}
\newcommand{\RegionMarkerTok}[1]{#1}
\newcommand{\SpecialCharTok}[1]{\textcolor[rgb]{0.00,0.00,0.00}{#1}}
\newcommand{\SpecialStringTok}[1]{\textcolor[rgb]{0.31,0.60,0.02}{#1}}
\newcommand{\StringTok}[1]{\textcolor[rgb]{0.31,0.60,0.02}{#1}}
\newcommand{\VariableTok}[1]{\textcolor[rgb]{0.00,0.00,0.00}{#1}}
\newcommand{\VerbatimStringTok}[1]{\textcolor[rgb]{0.31,0.60,0.02}{#1}}
\newcommand{\WarningTok}[1]{\textcolor[rgb]{0.56,0.35,0.01}{\textbf{\textit{#1}}}}
\usepackage{graphicx,grffile}
\makeatletter
\def\maxwidth{\ifdim\Gin@nat@width>\linewidth\linewidth\else\Gin@nat@width\fi}
\def\maxheight{\ifdim\Gin@nat@height>\textheight\textheight\else\Gin@nat@height\fi}
\makeatother
% Scale images if necessary, so that they will not overflow the page
% margins by default, and it is still possible to overwrite the defaults
% using explicit options in \includegraphics[width, height, ...]{}
\setkeys{Gin}{width=\maxwidth,height=\maxheight,keepaspectratio}
\IfFileExists{parskip.sty}{%
\usepackage{parskip}
}{% else
\setlength{\parindent}{0pt}
\setlength{\parskip}{6pt plus 2pt minus 1pt}
}
\setlength{\emergencystretch}{3em}  % prevent overfull lines
\providecommand{\tightlist}{%
  \setlength{\itemsep}{0pt}\setlength{\parskip}{0pt}}
\setcounter{secnumdepth}{0}
% Redefines (sub)paragraphs to behave more like sections
\ifx\paragraph\undefined\else
\let\oldparagraph\paragraph
\renewcommand{\paragraph}[1]{\oldparagraph{#1}\mbox{}}
\fi
\ifx\subparagraph\undefined\else
\let\oldsubparagraph\subparagraph
\renewcommand{\subparagraph}[1]{\oldsubparagraph{#1}\mbox{}}
\fi

%%% Use protect on footnotes to avoid problems with footnotes in titles
\let\rmarkdownfootnote\footnote%
\def\footnote{\protect\rmarkdownfootnote}

%%% Change title format to be more compact
\usepackage{titling}

% Create subtitle command for use in maketitle
\providecommand{\subtitle}[1]{
  \posttitle{
    \begin{center}\large#1\end{center}
    }
}

\setlength{\droptitle}{-2em}

  \title{Regresion}
    \pretitle{\vspace{\droptitle}\centering\huge}
  \posttitle{\par}
    \author{}
    \preauthor{}\postauthor{}
    \date{}
    \predate{}\postdate{}
  

\begin{document}
\maketitle

\hypertarget{tarea-3.}{%
\section{Tarea 3.}\label{tarea-3.}}

\hypertarget{completada-por-karen-bolanos-alfaro}{%
\section{Completada por: Karen Bolaños
Alfaro}\label{completada-por-karen-bolanos-alfaro}}

\hypertarget{regresion-lineal}{%
\section{Regresión lineal}\label{regresion-lineal}}

Análisis del Problema

El desempeño de un automóvil se puede medir de diferentes formas.
Algunas comunes son la cantidad de caballos de fuerza y el rendimiento
del mismo, que se puede resumir en cuantas millas puede recorrer el
automóvil por cada galón de combustible que consume. Para los clientes,
potenciales compradores de un automóvil, este rendimiento es importante
pues puede ayudar a tomar una decisión con respecto a cuál automóvil
comprar (si, por ejemplo, el cliente quiere un auto que rinda por muchas
millas y pueda economizar en la compra de combustible).

Desde este punto de vista, tanto a clientes como a fabricadores de
automóviles, les conviene entender cuál es la relación entre diferentes
características del automóvil y su rendimiento, pues el conocer estas
relaciones les puede ayudar a inferir cuál va a ser la eficiencia del
vehículo a partir de ver los valores de otras características. Para
fabricantes, puede ser importante conocer estas relaciones para saber
cómo hacer cada modelo más eficiente con respecto al anterior.

Entendimiento de los Datos

Con el fin de analizar y tratar de estimar las millas por galón de
diferentes modelos de automóviles, se trabajó con un conjunto de datos
que contiene 398 observaciones y 9 variables:

\begin{itemize}
\tightlist
\item
  mpg (millas por galón): numérica, con un rango de 9 a 46.60.
\item
  cyl (cilindraje): categórica ordinal, con valores posibles de 3, 4, 5,
  6 y 8.
\item
  disp (desplazamiento): numérica, con un rango de 68 a 455.
\item
  hp (caballos de fuerza): numérica, con un rango de 46 a 230 y 6
  valores faltantes.
\item
  weight (peso): numérica, con un rango de 1613 a 5140.
\item
  acc (aceleración): numérica, con un rango de 8 a 24.80.
\item
  model year (año): categórica, con 13 valores diferentes representando
  el año del automóvil.
\item
  origin (origen): categórica, 3 valores posibles: 1, 2, 3.
\item
  model name (nombre del modelo): categórica, con 305 posibles valores.
\end{itemize}

\begin{Shaded}
\begin{Highlighting}[]
\KeywordTok{library}\NormalTok{(readr)}
\KeywordTok{library}\NormalTok{(GGally)}
\end{Highlighting}
\end{Shaded}

\begin{verbatim}
## Loading required package: ggplot2
\end{verbatim}

\begin{verbatim}
## Registered S3 method overwritten by 'GGally':
##   method from   
##   +.gg   ggplot2
\end{verbatim}

\begin{Shaded}
\begin{Highlighting}[]
\KeywordTok{library}\NormalTok{(caTools)}
\KeywordTok{library}\NormalTok{(lessR)}
\end{Highlighting}
\end{Shaded}

\begin{verbatim}
## 
## lessR 3.8.8     feedback: gerbing@pdx.edu     web: lessRstats.com/new
## ---------------------------------------------------------------------
## 1. d <- Read("")           Read text, Excel, SPSS, SAS or R data file
##                            d: default data frame, no need for data=
## 2. l <- Read("", var_labels=TRUE)   Read variable labels into l,
##                            required name for data frame of labels
## 3. Help()                  Get help, and, e.g., Help(Read)
## 4. hs(), bc(), or ca()     All histograms, all bar charts, or both
## 5. Plot(X) or Plot(X,Y)    For continuous and categorical variables
## 6. by1= , by2=             Trellis graphics, a plot for each by1, by2
## 7. reg(Y ~ X, Rmd="eg")    Regression with full interpretative output
## 8. style("gray")           Grayscale theme, + many others available
##    style(show=TRUE)        all color/style options and current values
## 9. getColors()             create many styles of color palettes
## 
## lessR parameter names now use _'s. Names with a period are deprecated.
## Ex:  bin_width  instead of  bin.width
\end{verbatim}

\begin{Shaded}
\begin{Highlighting}[]
\KeywordTok{library}\NormalTok{(visdat)}
\KeywordTok{library}\NormalTok{(dplyr)}
\end{Highlighting}
\end{Shaded}

\begin{verbatim}
## 
## Attaching package: 'dplyr'
\end{verbatim}

\begin{verbatim}
## The following object is masked from 'package:GGally':
## 
##     nasa
\end{verbatim}

\begin{verbatim}
## The following objects are masked from 'package:stats':
## 
##     filter, lag
\end{verbatim}

\begin{verbatim}
## The following objects are masked from 'package:base':
## 
##     intersect, setdiff, setequal, union
\end{verbatim}

\hypertarget{ejercicios}{%
\section{Ejercicios}\label{ejercicios}}

\begin{enumerate}
\def\labelenumi{\arabic{enumi}.}
\tightlist
\item
  Cargue el archivo auto-mpg\_g.csv en una variable
\end{enumerate}

\begin{Shaded}
\begin{Highlighting}[]
\NormalTok{autos <-}\StringTok{ }\KeywordTok{read.csv}\NormalTok{(}\StringTok{'auto-mpg_g.csv'}\NormalTok{)}
\end{Highlighting}
\end{Shaded}

\begin{enumerate}
\def\labelenumi{\arabic{enumi}.}
\setcounter{enumi}{1}
\tightlist
\item
  Utilizando Ggpairs cree un gráfico de los atributos del dataset,
  observe las correlaciones entre atributos Hagamos un ggpairs con todas
  las variables del dataset. Para generar este grafico no se grafica la
  variable model.name porque supera la cardinalidad permitida de 15.
  Como es categorica no nos aporta valor y no la vamos a graficar.
\end{enumerate}

\begin{Shaded}
\begin{Highlighting}[]
\KeywordTok{ggpairs}\NormalTok{(autos[}\OperatorTok{-}\DecValTok{9}\NormalTok{], }\DataTypeTok{progress =}\NormalTok{ F)}
\end{Highlighting}
\end{Shaded}

\includegraphics{Regresion_files/figure-latex/unnamed-chunk-3-1.pdf}
Como se puede observar, tenemos varias variables categoricas que para
aplicar regresion lineal, no nos van a aportar mucho valor, por ejemplo:
origin, cyl, model.year, model.name

Hagamos un ggpair de nuevo pero sin esas variables categoricas: cyl,
model.year, model.name y origin Se observa que solo la variable acc
presenta una distribucion masomenos normal.

\begin{Shaded}
\begin{Highlighting}[]
\KeywordTok{ggpairs}\NormalTok{(autos[}\OperatorTok{-}\KeywordTok{c}\NormalTok{(}\DecValTok{2}\NormalTok{, }\DecValTok{7}\NormalTok{,}\DecValTok{8}\NormalTok{,}\DecValTok{9}\NormalTok{)], }\DataTypeTok{progress =}\NormalTok{ F)}
\end{Highlighting}
\end{Shaded}

\includegraphics{Regresion_files/figure-latex/unnamed-chunk-4-1.pdf}

Un vistazo rapido a las estadisticas de las variables del set de datos.
IMPORTANTE: no tenemos valores faltantes como NA, pero los tenemos en 0
para la columna hp, hay que hacer algo con ellos probemos con
eliminarlos a ver que pasa, si no los eliminamos o tratamos, no podemos
aplicar log sobre ellos.

\begin{Shaded}
\begin{Highlighting}[]
\KeywordTok{summary}\NormalTok{(autos)}
\end{Highlighting}
\end{Shaded}

\begin{verbatim}
##       mpg             cyl             disp             hp       
##  Min.   : 9.00   Min.   :3.000   Min.   : 68.0   Min.   :  0.0  
##  1st Qu.:18.00   1st Qu.:4.000   1st Qu.:104.2   1st Qu.: 75.0  
##  Median :23.00   Median :4.000   Median :148.5   Median : 92.0  
##  Mean   :23.56   Mean   :5.455   Mean   :193.4   Mean   :102.9  
##  3rd Qu.:29.00   3rd Qu.:8.000   3rd Qu.:262.0   3rd Qu.:125.0  
##  Max.   :47.00   Max.   :8.000   Max.   :455.0   Max.   :230.0  
##                                                                 
##      weight          acc          model.year        origin     
##  Min.   :1613   Min.   : 8.00   Min.   :70.00   Min.   :1.000  
##  1st Qu.:2224   1st Qu.:14.00   1st Qu.:73.00   1st Qu.:1.000  
##  Median :2804   Median :16.00   Median :76.00   Median :1.000  
##  Mean   :2970   Mean   :15.71   Mean   :76.01   Mean   :1.573  
##  3rd Qu.:3608   3rd Qu.:17.00   3rd Qu.:79.00   3rd Qu.:2.000  
##  Max.   :5140   Max.   :25.00   Max.   :82.00   Max.   :3.000  
##                                                                
##            model.name 
##   ford pinto    :  6  
##   amc matador   :  5  
##   ford maverick :  5  
##   toyota corolla:  5  
##   amc gremlin   :  4  
##   amc hornet    :  4  
##  (Other)        :369
\end{verbatim}

\begin{enumerate}
\def\labelenumi{\arabic{enumi}.}
\setcounter{enumi}{2}
\tightlist
\item
  Separe los datos en 2 conjuntos, uno de entrenamiento y otro de
  pruebas. Normalmente se trabaja utilizando un 70-80\% de los datos
  para entrenamiento y el resto para pruebas.
\end{enumerate}

Recuerde fijar una semilla para que el documento sea reproducible.

Pista:
\url{https://www.rdocumentation.org/packages/caTools/versions/1.17.1/topics/sample.split}

Voy a crear un dataset nuevo basado en el anterior donde elimino las
variables categoricas porque no me aportan valor y me dan problema en
los calculos y voy a eliminar las filas que tienen hp en 0.

\begin{Shaded}
\begin{Highlighting}[]
\NormalTok{autos_}\DecValTok{2}\NormalTok{ <-}\StringTok{ }\NormalTok{autos }\OperatorTok\StringTok{ }
\StringTok{    }\KeywordTok{select}\NormalTok{(mpg,disp,hp,weight,acc) }\OperatorTok\StringTok{ }
\StringTok{    }\KeywordTok{filter}\NormalTok{(hp }\OperatorTok{>}\StringTok{ }\DecValTok{0}\NormalTok{)}
\end{Highlighting}
\end{Shaded}

Veamos los tipos de datos y si falta alguno en el nuevo set de datos -
Observamos que todos los datos que tenemos son de tipo integer y que no
tenemos NA.

\begin{Shaded}
\begin{Highlighting}[]
\KeywordTok{vis_dat}\NormalTok{(autos_}\DecValTok{2}\NormalTok{)}
\end{Highlighting}
\end{Shaded}

\includegraphics{Regresion_files/figure-latex/unnamed-chunk-7-1.pdf}

\begin{Shaded}
\begin{Highlighting}[]
\CommentTok{# no tenemos valores faltantes en el training set}
\KeywordTok{sum}\NormalTok{(}\KeywordTok{is.na}\NormalTok{(autos_}\DecValTok{2}\NormalTok{))}
\end{Highlighting}
\end{Shaded}

\begin{verbatim}
## [1] 0
\end{verbatim}

Voy a repartir el 80\% de los datos al set de training y el 20\% al de
pruebas

\begin{Shaded}
\begin{Highlighting}[]
\KeywordTok{set.seed}\NormalTok{(}\DecValTok{17}\NormalTok{)}
\NormalTok{data_split <-}\StringTok{ }\KeywordTok{sample.split}\NormalTok{(autos_}\DecValTok{2}\NormalTok{, }\DataTypeTok{SplitRatio =} \FloatTok{0.8}\NormalTok{)}

\NormalTok{training_set =}\StringTok{ }\KeywordTok{subset}\NormalTok{(autos_}\DecValTok{2}\NormalTok{, data_split }\OperatorTok{==}\StringTok{ }\OtherTok{TRUE}\NormalTok{)}
\NormalTok{test_set  =}\StringTok{ }\KeywordTok{subset}\NormalTok{(autos_}\DecValTok{2}\NormalTok{, data_split }\OperatorTok{==}\StringTok{ }\OtherTok{FALSE}\NormalTok{)}
\end{Highlighting}
\end{Shaded}

Histograma de las variables del training set

\begin{Shaded}
\begin{Highlighting}[]
\KeywordTok{hist}\NormalTok{(training_set}\OperatorTok{$}\NormalTok{mpg)}
\end{Highlighting}
\end{Shaded}

\includegraphics{Regresion_files/figure-latex/unnamed-chunk-9-1.pdf}

\begin{Shaded}
\begin{Highlighting}[]
\KeywordTok{plot}\NormalTok{(}\KeywordTok{density}\NormalTok{(training_set}\OperatorTok{$}\NormalTok{mpg), }\DataTypeTok{main =} \StringTok{"Density Plot: mpg"}\NormalTok{, }\DataTypeTok{ylab =} \StringTok{"Frequency"}\NormalTok{)}
\end{Highlighting}
\end{Shaded}

\includegraphics{Regresion_files/figure-latex/unnamed-chunk-9-2.pdf}

\begin{Shaded}
\begin{Highlighting}[]
\KeywordTok{hist}\NormalTok{(training_set}\OperatorTok{$}\NormalTok{disp)}
\end{Highlighting}
\end{Shaded}

\includegraphics{Regresion_files/figure-latex/unnamed-chunk-10-1.pdf}

\begin{Shaded}
\begin{Highlighting}[]
\KeywordTok{plot}\NormalTok{(}\KeywordTok{density}\NormalTok{(training_set}\OperatorTok{$}\NormalTok{disp), }\DataTypeTok{main =} \StringTok{"Density Plot: disp"}\NormalTok{, }\DataTypeTok{ylab =} \StringTok{"Frequency"}\NormalTok{)}
\end{Highlighting}
\end{Shaded}

\includegraphics{Regresion_files/figure-latex/unnamed-chunk-10-2.pdf}

\begin{Shaded}
\begin{Highlighting}[]
\KeywordTok{hist}\NormalTok{(training_set}\OperatorTok{$}\NormalTok{hp)}
\end{Highlighting}
\end{Shaded}

\includegraphics{Regresion_files/figure-latex/unnamed-chunk-11-1.pdf}

\begin{Shaded}
\begin{Highlighting}[]
\KeywordTok{plot}\NormalTok{(}\KeywordTok{density}\NormalTok{(training_set}\OperatorTok{$}\NormalTok{hp), }\DataTypeTok{main =} \StringTok{"Density Plot: hp"}\NormalTok{, }\DataTypeTok{ylab =} \StringTok{"Frequency"}\NormalTok{)}
\end{Highlighting}
\end{Shaded}

\includegraphics{Regresion_files/figure-latex/unnamed-chunk-11-2.pdf}

\begin{Shaded}
\begin{Highlighting}[]
\KeywordTok{hist}\NormalTok{(training_set}\OperatorTok{$}\NormalTok{weight)}
\end{Highlighting}
\end{Shaded}

\includegraphics{Regresion_files/figure-latex/unnamed-chunk-12-1.pdf}

\begin{Shaded}
\begin{Highlighting}[]
\KeywordTok{plot}\NormalTok{(}\KeywordTok{density}\NormalTok{(training_set}\OperatorTok{$}\NormalTok{weight), }\DataTypeTok{main =} \StringTok{"Density Plot: weight"}\NormalTok{, }\DataTypeTok{ylab =} \StringTok{"Frequency"}\NormalTok{)}
\end{Highlighting}
\end{Shaded}

\includegraphics{Regresion_files/figure-latex/unnamed-chunk-12-2.pdf}

\begin{Shaded}
\begin{Highlighting}[]
\KeywordTok{hist}\NormalTok{(training_set}\OperatorTok{$}\NormalTok{acc)}
\end{Highlighting}
\end{Shaded}

\includegraphics{Regresion_files/figure-latex/unnamed-chunk-13-1.pdf}

\begin{Shaded}
\begin{Highlighting}[]
\KeywordTok{plot}\NormalTok{(}\KeywordTok{density}\NormalTok{(training_set}\OperatorTok{$}\NormalTok{acc), }\DataTypeTok{main =} \StringTok{"Density Plot: acc"}\NormalTok{, }\DataTypeTok{ylab =} \StringTok{"Frequency"}\NormalTok{)}
\end{Highlighting}
\end{Shaded}

\includegraphics{Regresion_files/figure-latex/unnamed-chunk-13-2.pdf}

\begin{Shaded}
\begin{Highlighting}[]
\CommentTok{# boxplot de todo el set de training}
\KeywordTok{boxplot}\NormalTok{(training_set)}
\end{Highlighting}
\end{Shaded}

\includegraphics{Regresion_files/figure-latex/unnamed-chunk-14-1.pdf}

\begin{Shaded}
\begin{Highlighting}[]
\CommentTok{# tenemos outliers en HP}
\KeywordTok{boxplot}\NormalTok{(training_set}\OperatorTok{$}\NormalTok{hp)}
\end{Highlighting}
\end{Shaded}

\includegraphics{Regresion_files/figure-latex/unnamed-chunk-14-2.pdf}

\begin{Shaded}
\begin{Highlighting}[]
\CommentTok{# tenemos outliers en acc}
\KeywordTok{boxplot}\NormalTok{(training_set}\OperatorTok{$}\NormalTok{acc)}
\end{Highlighting}
\end{Shaded}

\includegraphics{Regresion_files/figure-latex/unnamed-chunk-14-3.pdf}

\begin{Shaded}
\begin{Highlighting}[]
\KeywordTok{scatter.smooth}\NormalTok{(}\DataTypeTok{x =}\NormalTok{ training_set}\OperatorTok{$}\NormalTok{hp, }\DataTypeTok{y =}\NormalTok{ training_set}\OperatorTok{$}\NormalTok{mpg, }\DataTypeTok{main =} \StringTok{"MPG ~ HP"}\NormalTok{)}
\end{Highlighting}
\end{Shaded}

\includegraphics{Regresion_files/figure-latex/unnamed-chunk-15-1.pdf}

\begin{Shaded}
\begin{Highlighting}[]
\KeywordTok{scatter.smooth}\NormalTok{(}\DataTypeTok{x =}\NormalTok{ training_set}\OperatorTok{$}\NormalTok{disp, }\DataTypeTok{y =}\NormalTok{ training_set}\OperatorTok{$}\NormalTok{mpg, }\DataTypeTok{main =} \StringTok{"MPG ~ Disp"}\NormalTok{)}
\end{Highlighting}
\end{Shaded}

\includegraphics{Regresion_files/figure-latex/unnamed-chunk-16-1.pdf}

\begin{enumerate}
\def\labelenumi{\arabic{enumi}.}
\setcounter{enumi}{3}
\tightlist
\item
  Cree un modelo de regresion lineal utilizando el atributo mpg como la
  variable objetivo y en base a las correlaciones observadas en el
  gráfico del punto 2 escoja al menos dos atributos para usarlos como
  variables predictoras para el modelo.
\end{enumerate}

Pista:
\url{https://www.rdocumentation.org/packages/lessR/versions/1.9.8/topics/reg}

Nota: Al crear el modelo utilice el conjunto de datos de entrenamiento
definido en el punto 3.

Para esta paso mis variables predictoras del modelo seran: disp y hp

Aplicamos la formula con las variables predictoras en la regresion
lineal.

\begin{Shaded}
\begin{Highlighting}[]
\CommentTok{# reg(mpg ~ hp + disp, data = training_set)}
\end{Highlighting}
\end{Shaded}

\hypertarget{hagamos-regresion-lineal-con-el-comando-lm}{%
\subsection{hagamos regresion lineal con el comando
lm}\label{hagamos-regresion-lineal-con-el-comando-lm}}

\begin{Shaded}
\begin{Highlighting}[]
\NormalTok{linear_reg =}\StringTok{ }\KeywordTok{lm}\NormalTok{(mpg }\OperatorTok{~}\StringTok{ }\NormalTok{hp }\OperatorTok{+}\StringTok{ }\NormalTok{disp, }\DataTypeTok{data =}\NormalTok{ training_set)}
\KeywordTok{print}\NormalTok{(linear_reg)}
\end{Highlighting}
\end{Shaded}

\begin{verbatim}
## 
## Call:
## lm(formula = mpg ~ hp + disp, data = training_set)
## 
## Coefficients:
## (Intercept)           hp         disp  
##    37.42699     -0.05122     -0.04357
\end{verbatim}

\begin{Shaded}
\begin{Highlighting}[]
\KeywordTok{summary}\NormalTok{(linear_reg)}
\end{Highlighting}
\end{Shaded}

\begin{verbatim}
## 
## Call:
## lm(formula = mpg ~ hp + disp, data = training_set)
## 
## Residuals:
##     Min      1Q  Median      3Q     Max 
## -11.767  -3.196  -0.488   2.531  16.650 
## 
## Coefficients:
##              Estimate Std. Error t value             Pr(>|t|)
## (Intercept) 37.426987   0.840073  44.552 < 0.0000000000000002
## hp          -0.051222   0.015803  -3.241              0.00132
## disp        -0.043573   0.005798  -7.515    0.000000000000617
## 
## Residual standard error: 4.653 on 310 degrees of freedom
## Multiple R-squared:  0.6597, Adjusted R-squared:  0.6575 
## F-statistic: 300.4 on 2 and 310 DF,  p-value: < 0.00000000000000022
\end{verbatim}

Hagamos ahora la preduccion del modelo

\begin{Shaded}
\begin{Highlighting}[]
\NormalTok{predict_results <-}\StringTok{ }\KeywordTok{predict}\NormalTok{(linear_reg,  test_set)}
\NormalTok{preds_comparison <-}\StringTok{ }\KeywordTok{data.frame}\NormalTok{(}\KeywordTok{cbind}\NormalTok{(}\DataTypeTok{actuals =}\NormalTok{ test_set}\OperatorTok{$}\NormalTok{mpg, }
                                  \DataTypeTok{predicteds =}\NormalTok{ predict_results))}
\NormalTok{preds_comparison}
\end{Highlighting}
\end{Shaded}

\begin{verbatim}
##     actuals predicteds
## 2        15  13.724904
## 7        14   6.376134
## 12       14  14.416741
## 17       18  23.787483
## 22       24  28.154743
## 27       10  13.805786
## 32       25  27.637198
## 37       18  22.195911
## 42       12  11.518674
## 47       19  21.411599
## 52       30  29.699731
## 57       24  27.637198
## 62       13  13.724904
## 67       11   8.080116
## 72       15  16.497582
## 77       22  28.261824
## 82       23  27.229744
## 87       13  14.749335
## 92       13  14.039882
## 97       18  22.244814
## 102      26  30.844221
## 107      18  22.195911
## 112      19  27.757257
## 117      29  31.954171
## 122      24  26.520291
## 127      15  21.411599
## 132      16  21.411599
## 137      14  15.887561
## 142      26  29.205131
## 147      26  28.530910
## 152      18  21.155491
## 157      16  15.887561
## 162      18  22.757030
## 167      23  27.075387
## 172      24  27.273317
## 177      23  27.690738
## 182      25  28.223581
## 187      16  15.887561
## 192      24  24.563455
## 197      33  30.747108
## 202      30  29.563682
## 207      13  15.887561
## 212      13  14.749335
## 217      36  31.013875
## 222      16  16.143669
## 227      19  21.514042
## 232      29  29.205131
## 237      34  28.905450
## 242      22  28.306781
## 247      36  30.388557
## 252      21  23.846353
## 257      21  21.983376
## 262      18  18.910083
## 267      27  27.273317
## 272      24  27.273317
## 277      32  29.912266
## 282      20  22.708127
## 287      18  16.655885
## 292      32  29.912266
## 297      23  15.773767
## 302      32  30.393887
## 307      34  26.237534
## 312      28  26.237534
## 317      30  26.978274
## 322      28  25.251346
## 327      30  27.633495
## 332      24  29.254726
## 337      26  25.917226
## 342      35  30.824286
## 347      35  29.624871
## 352      32  28.879494
## 357      25  24.165034
## 362      18  23.269245
## 367      29  27.242031
## 372      31  29.978784
## 377      34  29.135601
## 382      38  21.657047
## 387      27  26.237534
## 392      31  28.041641
\end{verbatim}

\begin{enumerate}
\def\labelenumi{\arabic{enumi}.}
\setcounter{enumi}{4}
\tightlist
\item
  Realice predicciones utilizando el conjunto de pruebas y evalue el
  resultado con la métrica MSE.
\end{enumerate}

Pista:
\url{https://www.rdocumentation.org/packages/mltools/versions/0.3.5/topics/mse}

\begin{Shaded}
\begin{Highlighting}[]
\CommentTok{#library(Metrics)}
\KeywordTok{library}\NormalTok{(mltools)}

\CommentTok{# voy a predecir el error basada en mi nuevo dataframe preds_comparison}
\KeywordTok{mse}\NormalTok{(}\DataTypeTok{preds =}\NormalTok{ preds_comparison}\OperatorTok{$}\NormalTok{predicteds, }\DataTypeTok{actuals =}\NormalTok{ preds_comparison}\OperatorTok{$}\NormalTok{actuals)}
\end{Highlighting}
\end{Shaded}

\begin{verbatim}
## [1] 16.66146
\end{verbatim}

\begin{Shaded}
\begin{Highlighting}[]
\CommentTok{# aqui con log tenia 14.38}
\end{Highlighting}
\end{Shaded}

Obtenemos un MSE de 16.66146, sin aplicar ningun tratamiendo a las
variables predictoras ni a la independiente.

\begin{enumerate}
\def\labelenumi{\arabic{enumi}.}
\setcounter{enumi}{5}
\tightlist
\item
  Opcional
\end{enumerate}

6.a Pruebe varios modelos que utilicen diferentes variables y comparar
los resultados obtenidos

6.b Investigar como implementar en R las técnicas de preprocesado y
normalización vistas en clase y aplicarlas a los datos antes de pasarlos
al modelo.

\hypertarget{prueba-1-probando-con-hp-y-disp-aplicando-log-natural}{%
\section{Prueba 1: probando con hp y disp aplicando log
natural}\label{prueba-1-probando-con-hp-y-disp-aplicando-log-natural}}

El error ha bajado con respecto a la primera prueba sin aplicar log,
ahora tenemos el MSE en: 15.65637

\begin{Shaded}
\begin{Highlighting}[]
\NormalTok{linear_reg2 =}\StringTok{ }\KeywordTok{lm}\NormalTok{(}\KeywordTok{log}\NormalTok{(mpg) }\OperatorTok{~}\StringTok{ }\KeywordTok{log}\NormalTok{(hp) }\OperatorTok{+}\StringTok{ }\KeywordTok{log}\NormalTok{(disp), }\DataTypeTok{data =}\NormalTok{ training_set)}

\CommentTok{# calcular predicciones}
\NormalTok{results2 <-}\StringTok{ }\KeywordTok{predict}\NormalTok{(linear_reg2,  test_set)}
\NormalTok{predicted_}\DecValTok{2}\NormalTok{ <-}\StringTok{ }\KeywordTok{data.frame}\NormalTok{(}\KeywordTok{cbind}\NormalTok{(}\DataTypeTok{actuals =}\NormalTok{ test_set}\OperatorTok{$}\NormalTok{mpg, }
                                  \DataTypeTok{predicteds =} \KeywordTok{exp}\NormalTok{(results2)))}

\CommentTok{# calculo del error}
\KeywordTok{mse}\NormalTok{(}\DataTypeTok{preds =}\NormalTok{ predicted_}\DecValTok{2}\OperatorTok{$}\NormalTok{predicteds, }\DataTypeTok{actuals =}\NormalTok{ predicted_}\DecValTok{2}\OperatorTok{$}\NormalTok{actuals)}
\end{Highlighting}
\end{Shaded}

\begin{verbatim}
## [1] 15.65637
\end{verbatim}

\hypertarget{prueba-2-probando-con-weight-y-acc-en-su-estado-natural}{%
\section{Prueba 2: probando con weight y acc en su estado
natural}\label{prueba-2-probando-con-weight-y-acc-en-su-estado-natural}}

El error ha bajado con respecto a la ultima prueba, ahora tenemos el MSE
en: 15.03812

\begin{Shaded}
\begin{Highlighting}[]
\CommentTok{# aplicar modelo}
\NormalTok{linear_reg3 =}\StringTok{ }\KeywordTok{lm}\NormalTok{(mpg }\OperatorTok{~}\StringTok{ }\NormalTok{weight }\OperatorTok{+}\StringTok{ }\NormalTok{acc, }\DataTypeTok{data =}\NormalTok{ training_set)}

\CommentTok{# calcular predicciones}
\NormalTok{results3 <-}\StringTok{ }\KeywordTok{predict}\NormalTok{(linear_reg3,  test_set)}
\NormalTok{predicted_}\DecValTok{3}\NormalTok{ <-}\StringTok{ }\KeywordTok{data.frame}\NormalTok{(}\KeywordTok{cbind}\NormalTok{(}\DataTypeTok{actuals =}\NormalTok{ test_set}\OperatorTok{$}\NormalTok{mpg, }
                                  \DataTypeTok{predicteds =}\NormalTok{ results3))}

\CommentTok{# calculo del error}
\KeywordTok{mse}\NormalTok{(}\DataTypeTok{preds =}\NormalTok{ predicted_}\DecValTok{3}\OperatorTok{$}\NormalTok{predicteds, }\DataTypeTok{actuals =}\NormalTok{ predicted_}\DecValTok{3}\OperatorTok{$}\NormalTok{actuals)}
\end{Highlighting}
\end{Shaded}

\begin{verbatim}
## [1] 15.03812
\end{verbatim}

\begin{Shaded}
\begin{Highlighting}[]
\CommentTok{# aumenta el error usando weight con acc}
\end{Highlighting}
\end{Shaded}

\hypertarget{prueba-3probando-con-weight-y-acc-con-log}{%
\section{Prueba 3:probando con weight y acc con
log}\label{prueba-3probando-con-weight-y-acc-con-log}}

Aumenta el MSE: 16.32081

\begin{Shaded}
\begin{Highlighting}[]
\CommentTok{# aplicar modelo}
\NormalTok{linear_reg4 =}\StringTok{ }\KeywordTok{lm}\NormalTok{(}\KeywordTok{log}\NormalTok{(mpg) }\OperatorTok{~}\StringTok{ }\KeywordTok{log}\NormalTok{(weight) }\OperatorTok{+}\StringTok{ }\KeywordTok{log}\NormalTok{(acc), }\DataTypeTok{data =}\NormalTok{ training_set)}

\CommentTok{# calcular predicciones}
\NormalTok{results4 <-}\StringTok{ }\KeywordTok{predict}\NormalTok{(linear_reg4,  test_set)}
\NormalTok{predicted_}\DecValTok{4}\NormalTok{ <-}\StringTok{ }\KeywordTok{data.frame}\NormalTok{(}\KeywordTok{cbind}\NormalTok{(}\DataTypeTok{actuals =}\NormalTok{ test_set}\OperatorTok{$}\NormalTok{mpg, }
                                  \DataTypeTok{predicteds =} \KeywordTok{exp}\NormalTok{(results4)))}

\CommentTok{# calculo del error}
\KeywordTok{mse}\NormalTok{(}\DataTypeTok{preds =}\NormalTok{ predicted_}\DecValTok{4}\OperatorTok{$}\NormalTok{predicteds, }\DataTypeTok{actuals =}\NormalTok{ predicted_}\DecValTok{4}\OperatorTok{$}\NormalTok{actuals)}
\end{Highlighting}
\end{Shaded}

\begin{verbatim}
## [1] 16.32081
\end{verbatim}

\begin{Shaded}
\begin{Highlighting}[]
\CommentTok{# disminuye el error usando weight con acc con log}
\end{Highlighting}
\end{Shaded}

\hypertarget{prueba-4-probando-con-disp-y-weight-en-su-estado-natural}{%
\section{Prueba \#4: probando con disp y weight en su estado
natural}\label{prueba-4-probando-con-disp-y-weight-en-su-estado-natural}}

MSE: 15.50007

\begin{Shaded}
\begin{Highlighting}[]
\CommentTok{# aplicar modelo}
\NormalTok{linear_reg5 =}\StringTok{ }\KeywordTok{lm}\NormalTok{(mpg }\OperatorTok{~}\StringTok{ }\NormalTok{disp }\OperatorTok{+}\StringTok{ }\NormalTok{weight, }\DataTypeTok{data =}\NormalTok{ training_set)}

\CommentTok{# calcular predicciones}
\NormalTok{results5 <-}\StringTok{ }\KeywordTok{predict}\NormalTok{(linear_reg5,  test_set)}
\NormalTok{predicted_}\DecValTok{5}\NormalTok{ <-}\StringTok{ }\KeywordTok{data.frame}\NormalTok{(}\KeywordTok{cbind}\NormalTok{(}\DataTypeTok{actuals =}\NormalTok{ test_set}\OperatorTok{$}\NormalTok{mpg, }
                                  \DataTypeTok{predicteds =}\NormalTok{ results5))}
\CommentTok{# calculo del error}
\KeywordTok{mse}\NormalTok{(}\DataTypeTok{preds =}\NormalTok{ predicted_}\DecValTok{5}\OperatorTok{$}\NormalTok{predicteds, }\DataTypeTok{actuals =}\NormalTok{ predicted_}\DecValTok{5}\OperatorTok{$}\NormalTok{actuals)}
\end{Highlighting}
\end{Shaded}

\begin{verbatim}
## [1] 15.50007
\end{verbatim}

\hypertarget{prueba-5-probando-con-disp-y-weight-con-log}{%
\section{Prueba \#5: probando con disp y weight con
log}\label{prueba-5-probando-con-disp-y-weight-con-log}}

El MSE ha aumentado a 16.86141

\begin{Shaded}
\begin{Highlighting}[]
\CommentTok{# aplicar modelo}
\NormalTok{linear_reg6 =}\StringTok{ }\KeywordTok{lm}\NormalTok{(}\KeywordTok{log}\NormalTok{(mpg) }\OperatorTok{~}\StringTok{ }\KeywordTok{log}\NormalTok{(disp) }\OperatorTok{+}\StringTok{ }\KeywordTok{log}\NormalTok{(weight), }\DataTypeTok{data =}\NormalTok{ training_set)}

\CommentTok{# calcular predicciones}
\NormalTok{results6 <-}\StringTok{ }\KeywordTok{predict}\NormalTok{(linear_reg6,  test_set)}
\NormalTok{predicted_}\DecValTok{6}\NormalTok{ <-}\StringTok{ }\KeywordTok{data.frame}\NormalTok{(}\KeywordTok{cbind}\NormalTok{(}\DataTypeTok{actuals =}\NormalTok{ test_set}\OperatorTok{$}\NormalTok{mpg, }
                                  \DataTypeTok{predicteds =} \KeywordTok{exp}\NormalTok{(results6)))}

\CommentTok{# calculo del error}
\KeywordTok{mse}\NormalTok{(}\DataTypeTok{preds =}\NormalTok{ predicted_}\DecValTok{6}\OperatorTok{$}\NormalTok{predicteds, }\DataTypeTok{actuals =}\NormalTok{ predicted_}\DecValTok{6}\OperatorTok{$}\NormalTok{actuals)}
\end{Highlighting}
\end{Shaded}

\begin{verbatim}
## [1] 16.86141
\end{verbatim}

\hypertarget{prueba-6-probando-con-acc-y-hp-con-log}{%
\section{Prueba \#6: probando con acc y hp con
log}\label{prueba-6-probando-con-acc-y-hp-con-log}}

MSE aumento a 18.55948

\begin{Shaded}
\begin{Highlighting}[]
\CommentTok{# aplicar modelo}
\NormalTok{linear_reg7 =}\StringTok{ }\KeywordTok{lm}\NormalTok{(}\KeywordTok{log}\NormalTok{(mpg) }\OperatorTok{~}\StringTok{ }\KeywordTok{log}\NormalTok{(acc) }\OperatorTok{+}\StringTok{ }\KeywordTok{log}\NormalTok{(hp), }\DataTypeTok{data =}\NormalTok{ training_set)}

\CommentTok{# calcular predicciones}
\NormalTok{results7 <-}\StringTok{ }\KeywordTok{predict}\NormalTok{(linear_reg7,  test_set)}
\NormalTok{predicted_}\DecValTok{7}\NormalTok{ <-}\StringTok{ }\KeywordTok{data.frame}\NormalTok{(}\KeywordTok{cbind}\NormalTok{(}\DataTypeTok{actuals =}\NormalTok{ test_set}\OperatorTok{$}\NormalTok{mpg, }
                                  \DataTypeTok{predicteds =} \KeywordTok{exp}\NormalTok{(results7)))}

\CommentTok{# calculo del error}
\KeywordTok{mse}\NormalTok{(}\DataTypeTok{preds =}\NormalTok{ predicted_}\DecValTok{7}\OperatorTok{$}\NormalTok{predicteds, }\DataTypeTok{actuals =}\NormalTok{ predicted_}\DecValTok{7}\OperatorTok{$}\NormalTok{actuals)}
\end{Highlighting}
\end{Shaded}

\begin{verbatim}
## [1] 18.55948
\end{verbatim}

\hypertarget{prueba-7-probando-con-acc-y-hp-en-su-estado-natural}{%
\section{Prueba \#7: probando con acc y hp en su estado
natural}\label{prueba-7-probando-con-acc-y-hp-en-su-estado-natural}}

MSE: 17.00887

\begin{Shaded}
\begin{Highlighting}[]
\CommentTok{# aplicar modelo}
\NormalTok{linear_reg8 =}\StringTok{ }\KeywordTok{lm}\NormalTok{(mpg }\OperatorTok{~}\StringTok{ }\NormalTok{acc }\OperatorTok{+}\StringTok{ }\NormalTok{hp, }\DataTypeTok{data =}\NormalTok{ training_set)}

\CommentTok{# calcular predicciones}
\NormalTok{results8 <-}\StringTok{ }\KeywordTok{predict}\NormalTok{(linear_reg8,  test_set)}
\NormalTok{predicted_}\DecValTok{8}\NormalTok{ <-}\StringTok{ }\KeywordTok{data.frame}\NormalTok{(}\KeywordTok{cbind}\NormalTok{(}\DataTypeTok{actuals =}\NormalTok{ test_set}\OperatorTok{$}\NormalTok{mpg, }
                                  \DataTypeTok{predicteds =}\NormalTok{ results8))}

\CommentTok{# calculo del error}
\KeywordTok{mse}\NormalTok{(}\DataTypeTok{preds =}\NormalTok{ predicted_}\DecValTok{8}\OperatorTok{$}\NormalTok{predicteds, }\DataTypeTok{actuals =}\NormalTok{ predicted_}\DecValTok{8}\OperatorTok{$}\NormalTok{actuals)}
\end{Highlighting}
\end{Shaded}

\begin{verbatim}
## [1] 17.00887
\end{verbatim}

\hypertarget{prueba-8-probando-con-acc-y-hp-con-log-en-mpg}{%
\section{Prueba \#8: probando con acc y hp con log en
mpg}\label{prueba-8-probando-con-acc-y-hp-con-log-en-mpg}}

Aplicando log solo en la variable dependiente mpg obtengo el valor mas
bajo con MSE de 14.24683

\begin{Shaded}
\begin{Highlighting}[]
\CommentTok{# aplicar modelo}
\NormalTok{linear_reg9 =}\StringTok{ }\KeywordTok{lm}\NormalTok{(}\KeywordTok{log}\NormalTok{(mpg) }\OperatorTok{~}\StringTok{ }\NormalTok{acc }\OperatorTok{+}\StringTok{ }\NormalTok{hp, }\DataTypeTok{data =}\NormalTok{ training_set)}

\CommentTok{# calcular predicciones}
\NormalTok{results9 <-}\StringTok{ }\KeywordTok{predict}\NormalTok{(linear_reg9,  test_set)}
\NormalTok{predicted_}\DecValTok{9}\NormalTok{ <-}\StringTok{ }\KeywordTok{data.frame}\NormalTok{(}\KeywordTok{cbind}\NormalTok{(}\DataTypeTok{actuals =}\NormalTok{ test_set}\OperatorTok{$}\NormalTok{mpg, }
                                  \DataTypeTok{predicteds =} \KeywordTok{exp}\NormalTok{(results9)))}

\CommentTok{# calculo del error}
\KeywordTok{mse}\NormalTok{(}\DataTypeTok{preds =}\NormalTok{ predicted_}\DecValTok{9}\OperatorTok{$}\NormalTok{predicteds, }\DataTypeTok{actuals =}\NormalTok{ predicted_}\DecValTok{9}\OperatorTok{$}\NormalTok{actuals)}
\end{Highlighting}
\end{Shaded}

\begin{verbatim}
## [1] 14.24683
\end{verbatim}

\hypertarget{prueba-9-probando-con-disp-y-hp-con-log-en-mpg-y-hp}{%
\section{Prueba \#9: probando con disp y hp con log en mpg y
hp}\label{prueba-9-probando-con-disp-y-hp-con-log-en-mpg-y-hp}}

\begin{Shaded}
\begin{Highlighting}[]
\CommentTok{# aplicar modelo}
\NormalTok{linear_reg10 =}\StringTok{ }\KeywordTok{lm}\NormalTok{(}\KeywordTok{log}\NormalTok{(mpg) }\OperatorTok{~}\StringTok{ }\NormalTok{disp }\OperatorTok{+}\StringTok{ }\KeywordTok{log}\NormalTok{(hp), }\DataTypeTok{data =}\NormalTok{ training_set)}

\CommentTok{# calcular predicciones}
\NormalTok{results10 <-}\StringTok{ }\KeywordTok{predict}\NormalTok{(linear_reg10,  test_set)}
\NormalTok{predicted_}\DecValTok{10}\NormalTok{ <-}\StringTok{ }\KeywordTok{data.frame}\NormalTok{(}\KeywordTok{cbind}\NormalTok{(}\DataTypeTok{actuals =}\NormalTok{ test_set}\OperatorTok{$}\NormalTok{mpg, }
                                  \DataTypeTok{predicteds =} \KeywordTok{exp}\NormalTok{(results10)))}

\CommentTok{# calculo del error}
\KeywordTok{mse}\NormalTok{(}\DataTypeTok{preds =}\NormalTok{ predicted_}\DecValTok{10}\OperatorTok{$}\NormalTok{predicteds, }\DataTypeTok{actuals =}\NormalTok{ predicted_}\DecValTok{10}\OperatorTok{$}\NormalTok{actuals)}
\end{Highlighting}
\end{Shaded}

\begin{verbatim}
## [1] 14.48704
\end{verbatim}

\hypertarget{notas}{%
\section{Notas:}\label{notas}}

\begin{itemize}
\item
  solo la variable acc tiene distribucion normal, y como no la uso en mi
  reg lineal principal, entonces ninguna de mis variables tiene dist
  normal. Sera que eso me afecta en la prediccion?
\item
  Tenemos algunas variables categoricas, las debo quitar del data set?
  las debo quitar del training y test set?
\item
  mi primer mse fue de 31.81724, es demasiado
\item
  la primera vez que corri todo lo hice sin quitar de los dataset de
  prueba y training las variables categoricas, y entonces tenia errores,
  por ser categoricas, entonces las quité y corri todo de nuevo.
\item
  Que debo hacer para bajar ese error?? se me ocurre hacerle log a las
  columnas que estoy metiendo., o pasarlo por z-score
\item
  se me ocurrio hacer 6 columnas nuevas, aplicando log10 y log a las
  columnas que ocupo. Lo que no pense fue que como hp tiene 6 valores en
  0, entonces el log me da infinito, y eso revienta en la regresion
  lineal, entonces no me sirve. debo quitar esos valores en 0. como los
  quito?
\item
  quiere decir que debo preocuparme por los valores en 0 primero.
\item
  otra cosa que recomiendo es ordenar las columnas de menor a mayor para
  ver si hay valores muy extremos, de esa manera pude ver que tenia
  valores en infinito para hp, y que hp tiene valores en 0.
\item
  log 10 y log son muy parecidos, entonces me fui por log10 a las
  columnas que me interesan
\item
  la regresion la hago con todas las variables en log, pero luego, paso
  de vuelta los datos a notacion normal para que tenga mas sentido, y de
  esa comparacion es que saco el error.
\end{itemize}


\end{document}
